\title{This is a specimen title\tnoteref{t1,t2}} \tnotetext[t1]{This document is the results of the research
    project funded by the National Science Foundation.} \tnotetext[t2]{The second title footnote which is a longer
    text matter to fill through the whole text width and overflow into another line in the footnotes area of the first page.}

\author[1]{Jos Migchielsen\corref{cor1}% \fnref{fn1}}
\ead{J.Migchielsen@elsevier.com}
\author[2]{CV Radhakrishnan\fnref{fn2}} \ead{cvr@sayahna.org}
\author[3]{CV Rajagopal\fnref{fn1,fn3}} \ead[url]{www.stmdocs.in}
\cortext[cor1]{Corresponding author}
\fntext[fn1]{This is the first author footnote.} \fntext[fn2]{Another author footnote, this is a very long
    footnote and it should be a really long footnote. But this footnote is not yet sufficiently long enough to make two lines of footnote text.}
\fntext[fn3]{Yet another author footnote.}

\affiliation[1]{organization={Elsevier B.V.}, addressline={Radarweg 29},
    postcode={1043 NX}, city={Amsterdam}, country={The Netherlands}}

\affiliation[2]{organization={Sayahna Foundation}, addressline={JWRA 34, Jagathy},
city={Trivandrum}
postcode={695014},
country={India}}

\affiliation[3]{organization={STM Document Engineering Pvt Ltd.},
addressline={Mepukada, Malayinkil}, city={Trivandrum} postcode={695571},
Abstract
In this work we demonstrate ab the formation