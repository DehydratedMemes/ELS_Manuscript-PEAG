\begin{frontmatter}

    \title{Computational study of the glycolytic degradation of \iupac{poly|(ethylene terephthalate} catalyzed by \iupac{\N^1,\N^2-bis|(2-amino|benzyl)|-1,2-|diamino|ethane zinc (II)}}

    \author[1]{Pablo E. Alanis González\corref{cor1}}
    \cortext[cor1]{Corresponding author}
    \ead{pabloalanis1998@gmail.com}

    \author[1]{Isabel del Carmen Saenz Tavera}

    \author[1]{Victor Manuel Rosas García}

    \affiliation[1]{organization={Facultad de Ciencias Químicas, Universidad Autónoma de Nuevo León},
        addressline={Av. Universidad s/n, Cd. Universitaria},
        city={San Nicolás de los Garza},
        postcode={66455},
        state={Nuevo León},
        country={México}}

    \begin{abstract}
        A possible reaction mechanism of the glycolytic degradation of \iupac{poly|(ethylene terephthalate)} (PET) catalyzed with \iupac{\N^1,\N^2-bis|(2-amino|benzyl)|-1,2-|diamino|ethane zinc (II)} (ABEN) was determined using Kohn-Sham density functional theory (KS-DFT) using the range-separated hybrid, generalized gradient approximation functional, \chemomega B97X-V with DFT-D4 dispersion correction \cite{Mardirossian2014} using def2-TZVPP over the zinc, oxygen and nitrogen atoms and def2-SVP for the rest, Making use of an energy-weighted climbing image nudged elastic band (EW-CI-NEB) algorithm \cite{Asgeirsson2021} to determine the minimum energy path (MEP) and then optimizing the converged climbing image (CI) using eigenvector-following partitioned rational function optimization (EF P-RFO) to obtain the transition state (TS). The non-covalent interactions where obtained using and averaged independent gradient model (aIGM) algorithm \cite{Lefebvre2018}
    \end{abstract}

    %%Graphical abstract
    \begin{graphicalabstract}
        %\includegraphics{grabs}
    \end{graphicalabstract}

    %%Research highlights
    \begin{highlights}
        \item Research highlight 1
        \item Research highlight 2
    \end{highlights}

    \begin{keyword}

        %% Keywords
        KS-DFT \sep Polymer degradation \sep Catalysis

        % Codigos PACS
        \PACS 82.20.Pm \sep 82.35.-x \sep 82.65.Jn
    \end{keyword}

\end{frontmatter}